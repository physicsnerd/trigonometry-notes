\documentclass[12pt]{article}
\usepackage{graphicx}
\usepackage{amsmath}
\usepackage{float}
\usepackage[english]{babel}
\usepackage[english = american]{csquotes}
\MakeOuterQuote{"}
\usepackage[pdfpagemode=useNone,pdfstartview=FitH,colorlinks=true,linkcolor=blue,citecolor=blue,urlcolor=blue]{hyperref}
\usepackage[all]{hypcap}

\title{Single-Variable Calculus}
\author{physicsnerd}
\begin{document}

\maketitle
\tableofcontents

Trigonometry is a branch of mathematics used extensively in fields ranging from physics to higher level mathematics such as calculus. This will be a brief overview of some of the general principles of trigonometry.

\section{Degrees, Radians, and DMS}
There are multiple units that are used throughout trigonometry, ranging from degrees, which you are probably already familiar with, to radians and DMS (degrees, minutes, seconds). There are two main conversion rules to remember, then: $360^{\circ} = 2\pi \text{radians}$ and $1^{\circ}=60'=3600"$. What do these mean? Well first, remember that

\end{document}